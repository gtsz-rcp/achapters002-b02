% 이 문서는 B02의 소스코드이다.

\documentclass[a5paper,10pt, twoside, openright]{memoir}
\usepackage[hangul]{dhucs}
\usepackage{indentfirst} % 이 패키지를 불러오지 않을 경우 section 제목 다음 행의 문단은 들여쓰기를 하지 않는다. 
	\setsecindent{3em}  % 들여쓰기 너비는 3em으로 설정한다.

\begin{document}
			\frontmatter 	
			%책의 첫번째 부분이 시작된다. 책 제목과 작가 이름, 연도가 명시된 표지와 서문이다.
	
	\begin{titlingpage} %표지 \maketitle 이전에 표지에 나올 정보를 입력하고, \maketitle로 표지양식을 호출한다.	
			\title{B02}
			\author{미루}
			\date{2010}
		\maketitle
	\end{titlingpage}



		\chapter*{비공이} %서문
	이 책이 기획된 것은 2010년 여름 쯤이었고, 본격적으로 작업을 시작한 것은 그 해 거의 없다시피 했던 가을이 시작될 때 쯤 이었다. 이 책을 이루는 네 편의 글들은 2007년에서 2010년 사이 쓴 글들이다. 2010년의 나로선 실증할 순 없지만 소설이든 시든 뭐가 되든 문예로서 읽을 수 있는 것들이다. 이것이 이 책을 시작하고 마무리할 수 있는 결정적인 단서였다. 눈썰미 좋은 독자라면 소설적 태도와 시적 태도에 관한 고민의 흔적을 발견할 수 있을지 모르겠지만 없는 셈 치기로 하자.

	서문이 끝난 후 이어질 글들은 위태로운 구조 위에 헛도는 규칙에 의해 고장난 냉장고처럼 운영될 것이다. 그것은 허술한 서사로 보일 것이고, 실제로 그러할 것이다. 이 책의 차갑게 날 선 문장은 예리한 사유에서 비롯된 것이 아니다. 외풍이 심하기 때문이다. 어떤 빛나는 문장들은 진정성 있는 문제의식에 의한 것이 아니다. 당신 주변의 조명이 밝기 때문이다. 
	
	이 책은 인디 문화를 대변하지 않는다. 동시대 문예 경향과 관련 없다. 컨템포러리 아트와 관련 없다. 비전, 대안은 제시하지 않는다. 이 책의 내용은 세대적 비극과 무관하다. 오로지 개인의 불행으로 가득 차 있다. 때문에 뜻 모를 저주처럼 읽힐 수 있다. 이 책은 당신 주변의 이야기가 아니다. 무력한 한숨과 약자, 소수자에 대한 비아냥으로 가득 차 있다.

	하지만 혹시 누군가 이 허술한 문장들에서 예리한 질문이 떠올랐다면 함께 나누었으면 좋겠다. 어떤 질문들에 대해서는 흥분하며 대답할 것이고, 어떤 질문에 대해서는 전에 없던 온화한 표정으로 대답할 것이지만 대부분의 질문에 대해서는 당분간 우물쭈물하며 말을 더듬을 것이다. 하지만 언제까지 이 꼬라지일 수는 없을 것이다. 우리가 한 눈 파는 사이 언젠가 그 대답이 우리가 서 있는 바닥에 떨어져 있을지도 모른다. 혹시 누군가 이 엉성한 문장에 가슴을 찔렸다면 부디 많이 아프지 않길 바라겠다.

			\mainmatter %본격적인 책의 내용이 시작된다.
					\clearpage{}
				\setcounter{secnumdepth}{-2} 
					%본문 제목에 번호를 붙이지 않는다.
				\pagestyle{plain}
					%본문의 페이지 스타일을 설정한다.
					%기본값인 plain으로 설정했다.
				\pagenumbering{arabic}
					%본문의 페이지 번호 스타일을 설정한다.
					%아라비아 숫자로 페이지 번호를 표기한다.

			\chapter{석고보드}
	저쪽 석고보드는 오늘 아침부터 꿈틀거리는데 너넨 여태 모르는 모양이네. 어디 말이야? 저쪽 왼쪽 가장 먼 구석에서 내 쪽으로 아홉 번, 오른쪽으로 네 번 짚으면 닿을 석고보드. 새까맣게 부풀어 오르더니 이젠 꺼지는구나. 

	그날 갑자기 죽은 A가 석고보드에 목을 메달고 나를 내려다 본다. 다리 사이로 누런 똥오줌이 질척이고 이내 텅 빈 공책을 가득 채운다. 바로 전 페이지는 내 가래침과 끊어진 머리카락으로 가득 차 있다. 석고보드는 A의 처진 무게를 지탱하지 못하고 오래된 유리 땀처럼 흘러 내린다. 

	왼쪽 가장 먼 구석에서 내 쪽으로 아홉 번, 오른쪽으로 네 번 짚으면 나오는 석고보드가 산산조각나고 졸고 있던 B를 덮친다. B는 이미 축 쳐져 바닥으로 꺼졌다. B는 아래가 뚫린 머그컵으로 향하고 있다. 벌써 2년이 지났다. 

	매일 아침 A는 계단을 오른다.
	
	종종 계단에 놓인 젖은 빨래를 밟는다.

	옥상의 정상에서 A는 추락한다.

	매일 아침 A는 계단을 오른다. 

	B의 머리카락엔 탄피로 가득한 스크린세이버가 투사된다. 스크린세이버는 그가 졸겠다고 선언한 지 십오 분 뒤에 가동되었다. 스크린세이버는 매일 밤 샴푸를 한다. 잘 헹군 머리를 린스로 이 분간 마사지 한다. 단속적으로 B의 탄식이 천장으로 투사한다. \textbf{맙소사. 죽음마저 선언해야 한다니.}

	건물 1층의 제한된 화장실에는 세 가지 사건이 동시에 발생하고 동시에 마무리된다. 세 가지 사건은 서로의 사건에 개입하지 않는다. 제한된 화장실에서도 프라이버시는 존중되는 법이다. 
	
				\begin{adjustwidth}{1em}{3em}
					\begin{itemize}[\textperiodcentered]
						\item 보리차 티백이 가득한 좌변기 위로 단단하게 마른 생리대가 떨어진다. 변기 물 빠지는 소리가 들린다. 보리차 티백이 가득한 좌변기는 예전과 다름없이 파랗다. 
						\item 세 명의 여고생이 레종블루를 핀다. 한 여고생은 커버가 닫힌 변기 위에 앉아 발가락을 꼼지락 거린다. 한 여고생은 머리를 긁는다. 한 여고생은 고대기에 손가락을 데인다. 
						\item 대학생이 건물 소유주의 목을 조른다. 대학생은 아무 것도 아닌 말을 세 번 반복한다. 건물 소유주는 대학생에게 세 번 되묻는다. 대학생의 팔꿈치에 핏줄이 세 개 올라온다. 
					\end{itemize}
				\end{adjustwidth}
	
	이 방의 천장은 석고보드로 마감되어 있다. 석고보드 위 생수통 속의 고양이가 새끼를 낳는다. 행인의 손길이 닿은 새끼 고양이는 혀를 깨물고 생수통 속으로 들어간다. 머리만 내놓은 고양이는 결국 머리만 남는다. 그럼에도 눈꺼풀을 깜빡인다. 망가진 가로등 같이 띄엄띄엄 출산이 지속된다. 

	석고보드는 석고보드 아래로 길게 내려 앉는다. 우리는 결국 석고보드 바깥으로 밀려나 서 있을 수 밖에 없었다. 우리는 눈 앞의 천장에 직면해 있다. B의 한탄이 석고보드 아래 투사된다. 우리는 눈 앞의 한탄에 직면해 있다. 우리는 A가 죽음을 선언한 직후 A의 영정사진을 곱게 들고 서 있었다. 우리는 머리만 남은 고양이의 출산 건너편에 있다. A의 선언 전문은 작자 미상으로 이 방 정면에 매달린 현수막에서 찾아볼 수 있다. A는 작자 미상의 선언에 대한 보충 의견을 내놓기 위해 방문을 부수고 돌아간다. A는 제한된 화장실에서 보리차를 마시며 아직 끝나지 않은 의견을 낭독한다. 나머지 두 사건은 A에 어떤 개입도 하지 않는다. 변기 물 빠지는 소리가 들린다. 행인의 손이 닿은 고양이가 혀를 깨문다. 

	제한된 화장실에서도 프라이버시는 존중되는 법이다. 

	방문 한켠에 버려진 택배 상자가 있다. 구겨진 영수증이 있다. 단단한 석고보드가 우릴 침범한다. 우린 우릴 침범할 석고보드에 직면해 있다. 방 안은 천장으로 가득찬다. 우리 한 켠, 오늘부터 난 석고보드가 되겠다는 선언 첫 마디가 단속적으로 발행된다. 

	석고보드로 가득한 방 사이 단속적으로 A의 영정이 걸려있다. 

	난 왠지 외로워져 방 밖으로 나선다. 방문이 닫힌다. 방 입구에 작자 미상의 사망 선언이 내걸린다. 

	그때 그게 어떤 선언이든 이제 내게 단 한 문장의 의견도 남아있지 않았다. 

			\chapter{유예된 추락}
				\section{0}
	마지막 컬럼비아호의 마지막 폭파사고 최후의 생존자를 만나고 왔다. 그녀의 이름은 ↙였고, 지금은 ↘이다. 항상 그녀는 동세를 안고 있었고 지표면에 직각으로 치솟는 컬럼비아호에서 한순간 ↑였다.

	지구의 기류는 매 순간 달라지는 것이어서 폭파 이후 아직도 떨어지고 있는 그녀는 가끔 △이기도 하지만 그녀와 나의 대화에서는 뉴턴의 운동을 설명하는 도식처럼 왼쪽에서 오른쪽으로 흘러가야만 했다. 우리의 대화가 열어 놓은 마지막 컬럼비아호의 마지막 폭파를 진술하는 세계에서 그녀는 ↑이어야만 했다.

	컬럼비아호의 폭발은 그녀의 잘못이 아니었다. 그녀는 더듬더듬 폭발의 원인이 된 내열타일의 이름을 외웠다.

	대기권에 진입한 컬럼비아호는 그동안 아무런 문제도 되지 않았던 큰 문제에 직면했다. 대기였다. 대기는 늘 그랬던 것 처럼 컬럼비아호를 쓸어담았고, 컬럼비아호는 목적처럼 그를 뚫고 지나야만 했다. 그래야만 별빛을 실감할 수 있었고, 태양을 경이할 수 있었다.  그들은 그렇게 태어났던 것이다. 난 조용히 고개를 끄덕일 뿐이었다.

	부서진 내열타일의 틈새는 한순간에 컬럼비아호를 달구었다. 대기의 목덜미 쯤에서 컬럼비아호는 폭발했다.

	우린 컬럼비아호가 발사된 홍대 주변에서 긴 대화를 나눴다. 그녀가 말한 점멸하는 연애의 감정처럼 우리의 대화가 점멸하며 서로 몸을 부딪쳤다. 술취한 그녀의 한탄에서 아직 그녀의 동세가 살아 있다는 사실에 난 안도했다. 살아 남겨진 몸들에선 너무나 당연한 동세마저 박탈당하는 때가 많았기 때문이다.

	우린 잠시 유예된 것들에 대한 기쁜 대화를 나누었다. 그녀는 그녀의 동세를 내게 빌려주었다. 난 뉴턴의 세 가지 운동을 설명해 그녀를 조금 위로해주었다. 그녀는 나의 스물한 살을 담담하게 확인시켜주었다.

	중국 어딘가 있을 저기압 덩어리는 그녀가 막연히 부유에 몸을 내맡기게 하지 않았다.
	매번 그녀를 내쳤고, 그녀를 쓸어담았고, 비틀었다. 그녀는 중국에 있는 저기압을 의식하지 못하고 다만 불안해 할 따름이었다. 나 또한 중국 어디서 저기압이 어떤 자세로 무슨 표정을 지으며 그녀를 쳐다보는지 알 방법이 없었다. 중국 어딘가. 난 입을 다물고 말았다. 그 저기압이 그녀를 바라보고 있기나 한걸까 하는 생각이 들었기 때문이다.

	그녀보다 - 난 생각했던 것 보다 더 많이 불안해 하는 것이 분명했다.

	우린 많은 것을 바라지 않았다. 치명적인 생활 때문이었다. 그리고 중국 어딘가 있는 저기압 때문이었다. 그를 향해 달려가는 기류 때문이었다. 하얗게 달구어진 내열타일 때문이었다. 우린 그저 비명을 지르곤 했고, 우리가 할 수 있는 것의 목록을 써내려가야 할 베이지색 노트에 해야 할 것의 리스트를 날카롭게 파내야만 했다.

	우리의 20대는 이 모든 것들을 그냥 그렇다고, 어차피 그런거라고 무표정하게 대꾸할 틈을 주지 않았다. 우린 어떻게든 살고 싶었다. 살아야만 했다. 유예된 추락 이후에도.

				\section{1}
	그녀는 기류를 타고 있었으나 기류에 몸을 내맡기지는 못했다. 그녀가 먼지 섞인 땅에서 아픈 허리를 펴고 일어설 때 그녀와 나는 폐허가 된 홍대 발사장에서 기쁜 목소리로 그때의 추락을 나눌 것이다.
	
	나의 추락과 동세와 색감 또한.

	난 그녀에게 21세기가 시작될 무렵부터 되뇌어 온 말 하나를 선물해주었다. 난 익숙한 발음으로 ↘의 잘못이 아니에요. 내게 어떤 도움도 주지 못했던 이 말은 그녀에게 완성된 내열타일로 자리 잡기를 기대했다. 우린 얼굴이나 성기, 항문 등을 가릴 내열타일이 필요했을지도 모른다. 하얗게 달구어져 떨어져 나갈지도 모르겠지만. 부적 같은 것 말이다. 그녀는 맛있는 딸기쿠키를 선물했다. 정성들여 손으로 빚은 쿠키라는 말과 자기가 빚은 것이 아니라는 말을 덧붙이면서. 생각보다 맛있어서 눈을 동그랗게 뜨고 작별인사조차 하지 못했다.

	컬럼비아호가 발사된 홍대의 폐허는 많은 파일럿들을 끌어모았다. 그들은 가끔 몇 년 전의 20세기를 유언처럼 회상했다. 군중이 된 이야기 한복판에서 하늘을 바라보았다. 컬럼비아호의 폭발은 몇 년째 새빨갛게 계속되고 있었다. 폭발은 몇 년째 홍대를 울리고 있었다.

	딸기쿠키를 먹으며 돌아오는 긴 길 한복판에서 폭발이 지속되는 하늘을 보았다. 뭘 그리 서럽게 참았는지 온 세상에 악다구니를 쓰며 설치는 국지성호우가 지나가고 나면 다시 세상 모든 아스팔트는 희멀겋게 말라 들어갈 것이다. 그때 러시아의 미그 비행기는 투박하게 찢어진 구름 사이 노란 가루를 뿌리고 사라질 것이다. 언젠가 우린 장마를 아무렇지도 않게 대할 수 있을 것이다. 그녀와 나의 불안은 언젠가 기류를 타고 중국으로 넘어갈 것이다.



			\chapter{장마} 

	석 달 동안 A에게 전화하지 않았다. 그럼에도 A는 내 집에 찾아왔다. 난 죽어있는 듯 숨어있겠다고 다짐했고, 문이 부서져라 두드리는 A에게 대답조차 하지 않았다. A뿐 아니라 가스 검침원이 왔을 때도, 주인 아줌마가 왔을 때도, 취한 옆집 아저씨가 왔을 때도, 경찰 아저씨가 왔을 때도 대답하지 않았다. 

	A가 네번째로 찾아왔을 때는 내 무반응에 화가 났는지, 그치지 않는 비에 화가 났는지 애꿎은 우유 투입구를 부숴버렸다. 그 사이로 물에 젖은 십만원을 넣어두고 갔다. 그리고 다신 오지 않았다. 

	A가 돌아간 뒤 십만원을 꺼내고 빗물이 새어 들어오는 우유 투입구를 두루마리 화장지로 쑤셔 막았다. 오랫동안 밥을 먹지 않아 방에 누워있는데, 물이 가득 차 등에 욕창이라도 생기면 정말 희망이 없었기 때문이다. 욕창은 아스피린이나 타이레놀로 해결될 문제가 아니라는 것 쯤은 알고 있었다. 그걸로 해결될 수 없다면 방법은 없는 거다. 죽은 듯 처박혀 있었지만 죽긴 싫었다.

	석 달 동안 장판을 뜯어먹었다. 주인아줌마가 선심 쓰듯 깔아 준 노란 장판은 구워먹으면 별미였지만 석 달 전 우리집 가스는 끊겼다. 며칠 뒤 전기도 끊겼다. 

	석 달 전. 난 한 패밀리레스토랑에서 일했다. 그곳은 통조림에 든 음식 재료를 사는 것 보다 간판의 꺼진 형광등을 바꾸는 데 더 많은 돈을 쓰는 곳이었다. 난 가끔 통조림에 붙어있는 스티커를 뜯어 먹곤 했다. 다양한 종류의 통조림 스티커가 있었지만 그 중에서 유통기한이 오십 일 지난 아스파라거스 통조림의 스티커가 가장 맛있었다. 진실만을 말하건대, 그 통조림 스티커는 금방 익힌 아삭아삭 아스파라거스보다 훨씬 맛있었다. 

	비가 내리기 바로 전 날은 점장이 내게 약속했던 정직원 승진 날이었다. 정직원이 되면 내가 받던 월급보다 십오만 원 정도 더 받게 된다. 아침부터 기쁨에 가득 찬 난 이 년 오 개월 동안 모은 십이만 원으로 산 디올옴므st 정장을 입고 출근했다. 점장 사무실 문을 열었다. 점장은 몰래 빼돌린 후르츠칵테일 통조림을 먹으며 오스람 형광등 카탈로그를 읽고 있었다. 점장은 직원들에게 후르츠칵테일 통조림을 빼돌리면 통조림 따개로 대가리 뚜껑을 열어버리겠다고 협박하곤 했다. 우린 두려움에 떨며 후르츠칵테일을 냉장고에 따로 보관하기로 했다. 이 가게에선 머리 뚜껑이 열려도 산재 혜택을 받을 수 없었다. 

	한번은 어린 알바생이 배고픔을 이기지 못해 양송이 통조림을 따지도 않고 뱃속에 쑤셔 넣은 적이 있다. 우린 119도 부르지 못하고 그에게 까스활명수를 먹였다. 하지만 이 도시에 어울리지 않게 순박했던 그는 음식을 훔쳐 먹었다는 죄책감을 이기지 못하고 통조림 따개로 자기 머리를 열어버렸다. 오공본드로 대가리 뚜껑을 붙인 그는 아직도 이 레스토랑에서 일하고 있다. 

	우리 중 보험에 가입한 사람은 노망기가 있는 요리사 할아버지 밖에 없었다. 한번은 그가 심각한 표정으로 내게 아들이 내 이름으로 보험을 들어부렀다고 걱정한 적이 있다. 아저씨는 쥐도 새도 모르게 죽을 거니까 보험 회사에서 돈도 못 줄 거예요. 그는 나의 위로를 듣고 편안한 표정을 지었다. 
	
	점장이 형광등 카탈로그를 다음 날 장으로 넘겼다. 나는 그를 불렀다. 일어나서 한마디도 하지 않고 오다 보니 목소리가 갈라졌다. 점장 역시 갈라진 목소리로 가서 일 안 할거냐. 난 대가리 뚜껑이 열릴까봐 두려웠지만 그래도 말했다. 내가 정직원의 지읒을 발음하자마자 그는 내게 미안하다고 했다, 역시 갈라진 목소리로. 

	그 다음 날 일은 간단히 설명해야겠다. 난 조용히 가게 청소를 시작했다, 하이타이 냄새가 나는 촌스러운 알바생 유니폼을 입고 요리사와 캐셔가 오길 기다렸다. 30분후 그들이 도착하고 유니폼을 입고 조리실의 불을 올리는 순간. 물이 떨어지는 마대 자루와 물이 출렁이는 물통을 들고 테이블에 앉았다. 여기요! 커다란 목소리로 알바생을 부르고 신들린 듯 가게의 모든 메뉴를 읊었다. 모두 일흔 두 개의 메뉴였는데 그 중 서른 개는 네 명에서 다섯 명이 먹는 음식이었다. 다섯 시간이 지난 후 테이블에는 마지막에 먹은 You can't see California with Marlon Brando's Eyes with Toppokie(이탈리아식 전주비빔밥에 푸아그라를 첨가한 것)의 텅 빈 접시와 구토물이 담긴 스무 개의 양동이와 흔한 향수를 온몸에 적시고 DSLR로 날 찍는 뚱뚱한 여자와 날 개 쳐다보듯 하는 알바생과 역시 그런 눈으로 날 보는 점장이 있었다. 오른쪽에 한 뼘 정도 달아난 내 영혼이 있었다. 모두 칠십이만 육천사백사십 원이었다. 내겐 점장의 멱살을 잡고 받아낸 밀린 두 달치 임금이 있었다.

	그 이후의 기억은 응급실로 건너뛴다. 날 데리러온 A는 삼성의 입사 시험 최종 합격통지서와 계속 토하는 날 위해 가져온 검은 비닐 봉지를 들고 입원실 옆에 있었다. 의사는 이상한 언어로 진지하게 통보했다. 입원비가 없는 난 다음 날 날 포도당 링거를 맞다 쫓겨났고  집으로 돌아왔다. 

	공사 현장을 지나쳤다. 서울고시텔을 짓고 있었다. 내가 예전에 살았던 고시원 이름과 같았다. 그 고시원에서 육 년 동안 살았다. 육 년 동안 세 번 방값이 올랐고, 두 번 방이 좁아졌고, 세 명의 옆집 사람이 자살했다. 첫 번째 자살한 옆집 사람은 비쩍 마른 녀석이었다. 그 녀석은 얼핏 봐도 44사이즈의 옷이 몸에 헛돌 정도로 마르고 조그만 녀석이었다. 녀석은 꽤 많은 남녀와 밤을 지새우느라 방에 들어오지 않았고 가끔 자기 방에 데려와 밤을 지새우기도 했다. 그때마다 난 조용히 벽을 두드리며 그들의 흥분을 환기시키곤 했다. 그들의 사랑이든 육체든 시끄러우면 잠을 잘 수가 없었고, 그러면 알바하러 나가지도 못했기 때문이다. 우연히 알게 된 사실이지만 내 방과 녀석의 방을 나누는 벽은 연필 길이보다 더 얇았다. 

	내가 그녀를 마지막으로 본 건 그녀의 자살 소식을 듣기 몇 시간 전이었다. 그녀는 죽기 전날 일본, 베트남, 필리핀, 중국, 러시아 야동을 보며 밤을 새웠다. 다음 날 날 아침 고시원 주인은 투덜거리며 그 방의 짐을 모조리 뺐다. 누구도 그의 짐을 찾으러 오지 않았다는 게 이유였다. 

	두 번째 사람은 남자였다. 그는 안 좋은 냄새를 여기저기 듬뿍 풍기고 다니는 재주를 가지고 있었다. 그는 조용히 생활했고, 조용히 죽었다. 그의 소식을 듣게 된 건, 몇 주 후 나름대로 엄선한 향수를 겨드랑이에 뿌리는 논술교사가 그 방에서 살고 있다는 걸 알게 됐을 때였다. 그때 쯤 고시원이 리모델링을 했다. 난 한동안 A의 집에서 신세 지게 되었다. 다시 돌아간 고시원은 좀 더 좁아졌고, 벽은 더 얇아졌고, 방세는 좀 더 올랐다. 방이 좁아진 것은 어쩔 수 없다고 쳐도 방세가 오른 건 참을 수 없었다. 난 집주인에게 따졌다. 집주인은 평이한 표정으로 가스비가 올랐다고 했다. 난 조용히 방으로 돌아왔다. 

	그리고 몇 달 뒤 옆방의 논술교사가 고시원에 불을 질렀다. 논술교사는 벽을 뚫고 들어와 나를 밖으로 내보냈다. 여기저기 쓰레기가 많아 대청소를 해야겠단다. 고작 청소 따위에 심각한 표정을 짓는 그를 비웃으며 편의점으로 향했다. 편의점에는 내가 찾던 담배가 없었고, 국산 담배를 사들고 돌아온 나는 차마 집에 들어가지 못하고 불구경 하다 보험금을 받고 여기 집을 얻었다. 

	석 달 동안 비가 내렸다. 긴 장마 동안 넣어둔 양말 다섯 켤래와 팬티 두 벌, 티셔츠 여섯 벌은 파랗고 푹신푹신한 곰팡이로 덮였다. 

	석 달 뒤. 토 나올 정도로 내리던 장마는 멈췄다. 
	
	하수구마다 퉁퉁 분 사람이나, 주인아줌마, 개. 가끔 이구아나가 처박혀 있었다. 물이 빠지지 않았다. 창문 밖으로 넘실대는 빗물에 파란 곰팡이가 핀 옷들을 빨았다. 곰팡이가 핀 티셔츠는 니트 셔츠처럼 따뜻해보였지만 입을 순 없었다. 
	
	빨래타이는 내 옷을 하얗게 만들어 주었다. 거품과 함께 창문 밖 넘실대는 물을 파랗게 만들어 주었다. 
	
	창문 밖 조용히 한 사람이 떠다녔다. 난 그를 빗자루로 잡아주고 인사했다. 그는 얼굴도 팔도 손가락도 퉁퉁 부어올랐는데, 성기도 없고 가슴도 없어 마네킹인 줄 알았다. 그가 말하길 보여주기는 부끄럽지만 자기한테도 성기가 있다고 한다. 그녀는 여자였고, 성기가 없는 게 아니었다. 난 이 나이를 처먹고도 봉긋한 가슴이나 높은 목소리로 남녀를 구별했던 것이다. 그 여자는 내게 안부를 물어보았다. 붙임성 있는 사람이었다. 난 되는대로 내 안부를 지껄이고 그 여자에게 물어보았다. 요즘 어떻게 지내요? 그 여자는 아무 말 없이 물속에 잠겼다. 파란 물위에는 빨래타이가 내뿜는 기포만 가득했다.

	빨래를 끝내고 창문을 닫으려는 순간 비둘기가 날아왔다. 비둘기의 발에는 모나미 볼펜이 묶여있었다. 비둘기는 어이가 없다는 표정으로 바라보는 내게 종이를 달라고 말했다. 난 여전히 어이없는 표정으로 종이를 가져다주었다. 비둘기는 아주 능숙한 필기체로 글을 써내려갔는데, 내용은 이러하다. 

					\begin{quote}
						\textbf{비둘기가 쓴 편지}\\
						뭐함? 석 달 동안 방구석에쳐박혀있다니씨방새\\
						난 안보고 싶었니. 나도 보고 싶었어.\\
						비도 그쳤으니까 밖으로 나와\\
						돈도 없을 텐데 그냥 알아서 와\\
						방주로와\\
						기다릴게\\
						to A가 씀
					\end{quote}

	편지를 다 쓴 비둘기는 모나미 볼펜을 내팽개치고 날아갔다. 난 그제야 저 비둘기가 어떤 비둘기인지 알게 되었다. 그냥 비둘기였던 것이다. 실없는 농담 같지만 정말이다. 그냥 비둘기는 어쩌다보니 A를 지나치게 되었고 A의 간곡한 사연을 듣고 감동한 나머지 A의 말을 전하러 내게 온 것이다. 

	되는대로 지껄이던 A와 나는 언제부턴가 우리만의 모종의 언어를 사용하기 시작했다. 아주 새로운 언어라기보다는 한글로 쓸 수 있지만 어쨌든 우리만의 언어였다. 우린 언제부터 만났는지 말하지 않았다. 그리고 우리가 언제까지 만날 수 있는지도 말하지 않았다. 우리 언어로는 저런 말들을 할 수 없었다. 
	
	언젠가 A는 토익 문제집과 맨투맨을 들고 나를 찾아온 적이 있었다. 내게 토익 시험 준비를 같이 하자고 말했다. 난 천천히 그의 두꺼운 교재를 들어 올렸는데, 지져쓰. 내가 이십이 년간 읽었던 책 보다 더 두꺼웠고, 그걸 모두 합해도 이 책보다 더 글자가 많을 순 없었다. 게다가 난 fuck, ass, cock and shit이 없는 영어 문장은 상상조차 할 수 없었다. 그런 건 영어가 아니다. 
	
	난 A에게 네가 말한 토익 시험은 영어시험이 아닐 거라고 말했다. A는 어벙하긴 하지만 속은 치밀한 녀석이어서 그런 실수는 잘 하지 않는다. 하지만 원숭이도 나무에서 떨어질 때가 있다. A는 내 충격에 아랑곳하지 않고 책을 덥석 집어 들며 구백 점이면 삼성에 입사할 수 있다는 말을 했다. A는 삼성에 입사해서 매일 버거킹 와퍼세트를 먹는 꿈을 꾸고 있었다. 나 역시 그러고 싶었지만, 패밀리레스토랑에서 아르바이트를 하는 신세였고 A만큼은 매일 버거킹에서 와퍼를 먹었으면 좋겠다고 축복했다. 차마 아무도 쓰지 않는 언어를 위해 내 인생을 바칠 순 없었다. 그때까지만 해도 내 인생은 진절머리 나게 소중한 것이었다. 
	
	삼성은 이상한 곳이다. 어째서 아무도 쓰지 않는 언어를 사용하는 사람들을 신입사원으로 채용하는가. 내 상식으로는 전혀 이해할 수 없었다. A는 그 해 삼성 입사에 실패했다. 그가 알고 있었던 삼성에 입사하기 위한 토익 점수 최하점은 구백칠십 점이었다. A는 그것만 믿고 토익 시험에서 구백칠십 점만 맞췄다. 하지만 그 해 삼성 입사 자격은 구백팔십 점이었다. 난 자상하게도 A에게 이번엔 구백구십 점을 맞추라고 했고, 다음 날 해 A는 구백구십 점의 토익 성적표를 삼성에 재출했다. 

	내 목 언저리까지 차오른 빗물 속에도 버스는 여전히 움직이고 있었다. 놀라운 일이었다. 지하철은 승객들의 감전 위험이 있어 움직이지 않은 지 오래되었고, 비행기는 생각도 안 해봤고, 택시 역시 생각도 안 해봤다. 물 속에서도 버스는 움직이고 있었던 것이다. 내 앞으로 물살을 헤치고 버스가 멈춰 섰다. 난 아주 놀랐고 반가웠지만 내겐 버스비가 없었다. 하지만 난 왜인지 모르게 A를 만나야겠다고 생각했고 처음 만난 버스아저씨에게 사정했다. 아저씨는 조용히 고개를 끄덕였다. 

	라디오 방송에 귀를 기울였다. 이번 장마는 서울 지역에만 내렸다고 한다. 서울 외의 지역은 비가 한 방울도 내리지 않아 가뭄이 발생했단다. 심각한 물 부족으로 서울 시장은 선심 쓰듯 수도권의 전문대생을 고용해 서울을 둘러싸는 방벽을 만들었다고 한다. 그리고 물을 저장해 서울 외 지역에 나눠 준단다. 
	
	이어 아나운서는 다른 뉴스를 읊었다. 이번엔 물에 잠긴 서울대공원 소식이었다. 서울대공원도 홍수를 피해가지 못했다. 때문에 서울대공원 동물원의 짐승들은 다 죽었고, 기린과 티라노사우루스 밖에 안 남았다고 한다. 전문가는 기린과 티라노사우루스 중 티라노사우루스가 압도적으로 더 비쌌기 때문이라는 말은 하지 않았지만 희귀종이라는 이유로 기린보다 티라노사우루스를 먼저 구해야 한다고 말했다. 
	
	시청 앞 잔디광장에는 커다란 건축물이 세워졌다. 딱히 이름이 있었던 건 아니고, 있더라도 기억할 만큼 중요지도 않았다. 아나운서는 그걸 방주라고 불렀다. A가 말했던 방주는 아마 그 방주일 것 같다. 비가 내리기 시작할 때쯤 방주 근처에 많은 청년들이 모였다고 한다. 
	
	청년들은 한때 창녀였고, 소년이었고, 소녀였고, 군인이었고, 사이보그였다. 청년들은 거기 죽치고 앉아 술 마시거나 책 읽거나 파티를 하거나 난교를 하거나 울거나 감기 걸리거나 산책을 하거나 음악을 듣거나 노래를 하거나 밥을 먹었다. 그러다 갑자기 비가 쏟아지기 시작했고 물이 들어차는 와중에도 청년들은 방주가 있는 잔디광장에서 저런 일들을 했다. 그러다 다 죽은 것이다. 한 할아버지는 인터뷰에서 저들에 대해 죽어도 싸다는(말은 절대로 하지 않았다. 내가 느끼기에 그랬다. 억하심정일 수도 있다. 억하심정이 분명하다) 듯 침착하게 말했다. 지금도 시청 앞 잔디광장에는 수백 구의 익사체가 둥둥 떠다닌다고 한다. 아나운서는 그 익사체를 치우기 위해 천문학적인 돈이 들 것이며 커다란 납골당에 쑤셔놓고 태워버릴거라고 했다. 
	
	63빌딩 근처에서 버스 광고가 나왔다. 수영장 광고였다. 목소리가 예쁜 여자는 하늘하늘한 웃음을 날리며 자긴 63빌딩 옥상에서 수영을 할 거란다. 멋진 계획이다. 내릴 사람이 없어 버스는 정류장을 지나쳐가고 뉴스는 계속되었다. 코엑스 아쿠아리움의 물고기들이 모두 탈출했다는 소식에 이어 63빌딩에서 상어가 탈출했다는 소식을 전했다. 버스 옆으로 상어가 스쳐지나갔다. 

	반포대교를 지나가는 버스 아래도 물이 흐르고 있었다. 그리고 버스는 멈추고 말았다. 아저씨는 말없이 내려 버스를 앞으로 밀기 시작했다. 한참이 지났지만 버스는 꿈쩍도 하지 않았다. 나도 조용히 내려 아저씨 옆에서 버스를 밀었다. 역시 버스는 움직이지 않았다. 우리는 한참 동안 움직이지 않는 버스 뒤에서 쌩쑈를 했다. 그러다 아저씨는 내게 먼저 가라고 말했다. 하지만 난 그럴 수 없었다. 난 반포대교에서 시청 광장으로 어떻게 가는지도 모르고, 이 재난 속에서 혼자 움직일 수도 없었다. 하지만 아저씨는 막무가내였다. 이번엔 나를 밀어내기 시작했다. 그리고 손짓으로 이쪽을 향해 쭉 가면 시청역이 나올 거라고 말했다. 난 어쩔 수 없이 아저씨에게 작별인사를 하고 그가 손짓한 이쪽을 향해 걷기 시작했다. 

	생각보다 먼 거리였지만, 생각보다 위험하진 않았다. 난 반포대교를 통과할 때 오줌을 쌌고, 자유형을 하다 배영을 하며 시청광장을 향해 갔다. 사람은 아무도 없었다. 다행이었다. 

	버터플라이를 하며 도착한 시청 앞 광장에는 예상대로 익사체들이 떠 있었다. 그들은 죽기 전에도 그랬던 것처럼 떠다니며 술 마시거나 책 읽거나 파티를 하거나 난교하거나 울거나 감기 걸리거나 산책을 하거나 음악을 듣거나 노래를 하거나 밥을 먹고 있었다. 
	
	A는 방주의 문을 열고 시체들을 헤치며 내게 다가오고 있었다. A는 고무튜브를 허리에 끼우고 있었다. A는 내게 잘 지내느냐고 물었다. 충격이었다. 그녀가 내게 한국말로 안부를 물은 것이다. 슬펐다. 그러면서 마음이 놓였다. 난 그녀에게 되는대로 한국말과 우리만의 언어로(아니 어쩌면 지금은 나만의 언어로) 나의 안부를 지껄였다. A는 묻지도 않았지만 알아서 자기 안부를 말했다. 자긴 삼성에서 잘 지낸다고, 매일 버거킹 와퍼세트를 Large로 먹는다고, 매일 맥심 원두커피를 마신다고. 난 그냥 A가 말하는 걸 듣고 있었다. A는 계속 자기 안부를 말했고, 난 그냥 들으며 고개를 끄덕였다. 시청 앞 익사체의 군중에서 콘돔 하나가 떠다녔다. 난 그 콘돔을 주웠다. A와 나는 한때 생활고에 시달릴 정도로 콘돔을 많이 샀다. A는 내게 원망스러운 목소리로 다시 그때로 돌아가고 싶다고 말했다. 난 대답했다. 

	그때가 기억나지 않는다고.
	
	A에게 작별인사를 했다. A는 내게 작별인사를 하며 십만원을 주었다. 그때 난 돈이 한 푼도 없었기 때문에 그냥 조용히 돈을 받아들었다. 그리고 A와 난 헤어졌다. 

	A는 튜브를 타고 방주로 들어갔다. 방주의 문이 닫히고 큰 소리를 내며 방주가 하늘로 떠오르기 시작했다. 커다란 해일이 일어 익사체들을 덮쳤다. 난 끝까지 서서 익사체들과 떠오르는 방주를 바라보았다. 하늘엔 해가 떠 있었다. 

	목수들과 웨이트리스들과 마약중독자와 비서들과 알콜 중독자들과 신경쇠약자들과 소년들에게 키스하는 소년들과 소녀들에게 키스하는 소녀들과 소년들에게 키스하는 소녀들과 그 사이의 모든 청년 시체들은 서로의 몸을 끌어안고 섹스하기 시작했다. 그들은 각자 맘에 드는 속도로, 체위로, 목소리로, 이게 마지막이라는 듯 열심히 서로를 끌어안았다. 어느 순간 그들은 갑자기 노래를 부르기 시작했는데 다른 목소리로, 높낮이로, 빠르기로, 박자로 노래하기 시작했다. 
	
	난 저들 하나하나를 뚫어지게 쳐다보았다. 누군가와는 눈이 마주쳤고 누군가와는 스쳐지나갔고 누군가는 내가 언젠가 본 사람이었고 누군가는 내가 전혀 모르는 사람이었고 누군가는 날 찬 년이었고 누군가는 내가 찬 여자애였고 누군가는 알고보니 사촌동생이었다. 갑자기 코끝이 찡해졌다. 석 달 만에 본 햇빛이 내 볼을 태울 듯 했다. 난 조용히 손가락으로 코를 쥐고 잔디가 밟히는 바닥으로 몸을 숙였다. 
	
	그들은 하나같이 콘돔을 끼고 있었다. 난 코를 틀어쥐고 주머니 속 콘돔을 찾아보았다. 콘돔은 여전히 있었다. 그들에게 콘돔은 성기에도 있었고, 손가락에도 있었다. 콘돔은 그들의 얼굴을 감쌌고, 귓볼을 감쌌고, 발가락을 감쌌고, 배꼽을 덮었다. 물속에서도 노래는 들렸다. 그들은 어느 순간 입을 맞춰 말했다. 놀랄 것 없다고.

	내가 물 밖으로 고개를 내밀 때, 그들은 각자의 성기에서 콘돔을 빼내어 목까지 차오른 물에 깨끗이 씻고 있었다. 그리고 콘돔에 바람을 불어넣기 시작했다. 바람이 가득 찬 색색깔의 콘돔이 하늘로 날아가기 시작한다. 서울 하늘에 콘돔 풍선이 가득 차 있다. 그 위로 방주는 올라간다. 잔디광장에서 분수가 솟구쳐 나온다. 시청 앞 광장에 물결이 인다. 청년 익사체들이 떠다닌다. 물결에 맞춰 흩어진다. 침과 정액이 떠다닌다. 내 몸이 잠긴다. 석 달 만에 본 햇빛이 내 볼을 태울 듯 하다.  
	
	

			\chapter{발인}

	점심이었다. 사람들은 모두 나가고 혼자 사무실을 지키고 있었다. 며칠째 멈춰 있던 핸드폰이 진동하며 아버지의 부고를 알렸다. 핸드폰은 아직 차가웠다. 납작한 스크린이 어머니의 통곡을 전했다. 배달 받은 돌솥비빔밥은 식어갔다. 담담한 손으로 눌러찍은 문자는 여동생의 것이었다.

	내일 오전 발인. 성가롤로 병원 정문에서 문자해.

	아버지가 죽었다.

	백여든 번째 자살시도. 열다섯 번째 성공이다. 집에 있는 동안 아버지의 임종을 지켰으니 이번에는 죄책감 없이 발인을 지켜볼 생각이다.

	(아버지가 내게 죄책감을 요구한 것은 아니다. 그저 그런 책의 추천사처럼 죄책감이 달라붙어 있었다. 없어도 상관 없지만 있어서 손해볼 것은 없다는 마음으로 나는, 아버지는 당신의 자살을 감당했다. 실은 그게 있어 조금 나았다.)

	아버지의 자살은 필연적이었다.

	내가 처음 말을 할 무렵이었다. 아침 어머니는 정성껏 도루코 아이리스 식도를 갈았다. 안방의 원목 장롱은 항상 질긴 비니루로 꼼꼼히 덮여 있었다. 안방 한 가운데 메달린 아버지는 필사적으로 파란 손가락을 움직였다. 부푼 손가락이 손톱을 밀어 올렸다. 아빠. 미끄러운 비니루를 조심히 밟으며 아버지의 손가락을 쥐었다. 아빠. 그날 어머니는 내가 처음 말을 한 기념으로 아버지를 토막냈다.

	이후 나의 유치원 입학 기념으로 아버지를 토막냈다. 동생을 임신한 기념으로 아버지를, 동생이 태어난 기념으로, 나의 초등학교 입학 기념, 아버지의 승진, 세기말, 외환위기, 새로운 대통령, 월드컵, 또 다른 대통령, 할머니의 환갑, 나의 대학 입학, 동생의 중간고사가 있는 날 아침 어머니는 정성껏 도루코 아이리스 식도를 갈았다. 안방의 원목 장롱은 질긴 비니루로 꼼꼼히 덮여 있었다.

	내가 고등학교에 입학할 무렵 아버지는 조심스럽게 비니루를 딛으며 걸어오는 내게 미안하다고 말했다. 미안하다. 아버지가 뭘요. 괜찮아요. 그래도 미안하구나. 괜찮다고요. 내가 해준게 없구나. 괜찮다니까요. 미안하다. 정말 내게 너한테 미안한게 많다. 미안해.

	그날 난 어머니를 도와 도루코 아이리스 식도를 갈았다. 숫돌이 새빨갛게 달아올랐다.

	소파 한 구석에서 문제집을 풀던 여동생이 입안에서 순결캔디를 달그락 거리며 어머니에게 물었다. 엄마는 왜 칼을 가는데? 쓸 데가 있어. 왜? 쓸 데가 있다고. 그게 뭔데? 쓸 데가 있다니까. 엄마, 왜? 니네 아빠를 죽여야 해. 니네 아빠는 나쁜 사람이야. 그래서 니네 아빠는 죽어야 해. 이후 여동생은 내게도, 아버지에게도, 어머니에게도 아무 말도 하지 않았다.

	톨게이트를 한 번 지나 국도를 타고 한참 가야 나오는 저수지에 차를 세웠다. 이제 막 리스가 끝난 어머니의 현대차였다. 트렁크 가득 찬 종량제 봉투를 바케스에 나눠 담았다. 시멘트가 굳길 기다리며 차에 들어가 히터를 켰다. 그때 처음 아버지로부터 전화가 왔다.

	미안하다. 아버지는 달아오른 차 창에 하얗게 김이 서릴 때까지 내게 미안하다. 시멘트가 굳으며 아버지의 안구가 귤빛으로 마를 때까지 내게 미안하다 말했다.

	아버지는 집과 차와 회사와 룸싸롱을 오가며 자살하기 시작했다. 어머니가 도루코 아이리스 식도를 정성껏 갈고 있을 때 혀를 깨물고 자살했다. 어머니는 아버지의 입에 주먹만한 재갈을 쑤셔 넣었다. 호흡을 멈추고 자살했다. 산소호흡기가 달린 주먹만한 재갈을 쑤셔 넣었다. 굶어 자살했다. 아버지 옆에 포도당 수액이 가득 든 드럼통이 준비되었다.

	아버지는 어머니에게 묶이기 전에 자살했다. 어머니가 화장실에 간 사이 자살했다. 토막날 때 자살했다. 칼을 가는 동안 자살했다. 매장될 때 자살했다.

	어머니는 아버지를 더 단단히 묶었다. 더 많은 수액을 준비했다. 아버지의 자살시도는 계속 되었다. 어머니는 매순간 아버지의 자살을 막으려 했다. 그리고 연구했다. 부엌의 칼꽃이에는 도루코 아이리스 식도가 꽂혀 있었다.

	자살한 아버지의 첫번째 장례식을 준비하러 가는 길. 나는 트렁크에 묻은 시멘트 가루를 보며 그 동안 아버지가 왜 자살시도를 했는지 이해할 수 있었다. 내 옆에 멈춰선 어머니는 습관처럼 흐느꼈다.

	언젠가 이런 일도 있었다. 수퍼에 들러 담배를 사오는 길 어머니와 마주쳤다. 어머니는 리스 기간이 얼마 안 남은 현대차를 타고 있었다. 트렁크가 닫히지 않았는지 덜컹거렸다. 안전턱을 지나던 중 트렁크가 활짝 열렸다. 아버지의 다리가 구겨져 담긴 종량제 봉투가 트렁크 밖으로 떨어졌다. 같은 시각 중간고사를 마친 여동생과 마주쳤다. 교복 치마 아래 쳐진 체육복 바지를 입고 있었다. 난 봉투를 집어들어 트렁크에 던져 넣었다. 아버지의 머리가 담긴 종량제 봉투가 내게 미안하다 말했다. 트렁크 문이 잘 닫힌 걸 확인한 어머니가 차를 몰고 아파트를 빠져나갔다.

	토막난 아버지의 주검은 단지 내 쓰레기 수거함에 있었다. 인근 야산에서 등산객에 의해 발견되었다. 대로변 가로수길에 묻혀 있었다. 재개발 중인 아파트 공사단지에서 인부들에 의해 발견되었다. 저수지에 버려졌다. 달동네 배고픈 고양이들의 먹이가 되었다.

	그날 저녁 퇴근 시각이 지나 리모콘으로 차 문 잠그는 소리가 들렸다. 옥상으로 향하던 엘리베이터가 집 앞에서 멈췄다. 아버지는 늘 그렇게 살아 돌아왔다. 그것 말고는 달리 할 수 있는 일이 없었다.

	아버지의 장례를 치르던 중 아버지로부터 전화가 왔다. 미안하다. 아버지는 반복했다. 미안하다. 아버지는 내게 문자를 보냈다. 미안하다. 이 반복되는 진심에 무뎌갈 때쯤 미안하다. 이메일로, 미니홈피 방명록에, 트위터 멘션으로 미안하다. 아버지는 어떻게 해서든 당신이 나와 접속할 수 있다는 걸 보여주고 싶었던 것이다. 장례식장엔 도루코 아이리스 식도가 없었다. 여동생은 문제집을 풀고 있었다.

	장례식이 끝난 저녁 퇴근 시간이 지나 리모콘으로 차 문 잠그는 소리가 들린다. 옥상으로 향하던 엘리베이터가 집 앞에서 멈춘다. 아버지는 늘 그렇게 살아돌아왔다.

	여동생은 항상 순결캔디를 먹고 있었다. 빨고 핥고 깨물어 먹었다. 문제집 살 돈을 받을 때도, 어머니께 반항할 때도, 아침을 거를 때도, 생리중일 때도, 임신중절을 고민할 때도, 중간고사를 봤을 때도, 잘 봤을 때도, 임신중절을 했을 때도, 왠지 서러웠을 때도 여동생은 순결캔디를 먹고 있었다.

	집안 곳곳 문제집을 풀고 있는 동생이 앉아 있었다. 내가 몰래 야동을 볼 때 동생은 한 쪽 구석에서 문제집을 풀고 있었다. 엄마가 도루코 아이리스 식도를 갈고 있을 때, 아버지의 주검을 해체할 때, 아버지가 자살을 선택할 때, 내가 몰래 여자친구를 데려올 때, 아버지를 매장할 때, 아버지가 살아돌아왔을 때도 여동생은 한 쪽 구석에서 문제집을 풀고 있었다. 여동생은 그렇게 조용히 앉아 모든걸 듣고 있었다.

	식장 한 가운데서 고스톱을 치던 무리가 잠들자 어머니가 내게 말했다. 너도 알다시피 네 아버지는 좋은 사람이었다, 산 사람은 살아야지. 할머니가 말했다. 너도 알다시피 모두 네 애미 탓이다, 산 사람은 살아야지. 할아버지가 말했다. 너도 알다시피 사회를 선도하는 훌륭한 사람이 되어라, 산 사람은 살아야지. 네 어머니, 할머니, 할아버지. 여동생은 말없이 문제집을 풀고 있었다.

	장례가 모두 끝났다. 어머니는 아버지의 유골 단지를 끌어안고 있었다. 여동생은 마지막 한 문제를 남겨두고 있었다. 그날 저녁 퇴근 시간이 지나 리모콘으로 차 문 잠그는 소리가 들렸다. 옥상으로 향하던 엘리베이터가 집 앞에서 멈췄다. 아버지는 돌아오지 않았다. 그것 말고는 달리 할 수 있는 일이 없었다.

	아버지는 룸싸롱 화장실 변기에 고개를 쳐박고 있었다. 아버지의 허파에는 수돗물이 가득 차 있었다. 여동생과 나는 버지를 삼킨 화장터의 화로에서 물 끓는 소리를 들었다.

	요즘 어머니는 온종일 TV을 켜놓는다. 거실 소파에 누워 잔다. 열 다섯 개의 유골 단지를 번갈아가며 안고 있다. 여동생은 안방의 원목가구를 밖으로 빼냈다. 방 벽면에 꼼꼼히 세라믹 타일을 붙여 놓았다. 천장 귀퉁이마다 비니루를 쉽게 교체할 수 있도록 집게를 설치했다. 바닥 수채구멍엔 역류방지장치가 설치되었다. 천장 가운데 쇠사슬이 단단히 고정되어있다. 마지막 한 문제를 풀고 중간고사를 끝낸 여동생은 매일 아침 도루코 아이리스 식도를 간다. 엄마의 칼 가는 소리가 항상 거슬려왔단다. 난 그냥 집으로 돌아왔다.

	아버지가 죽었다.



			\chapter{조난} 
				\section{알립니다} 
				이 책의 저자인 저는 2019년 9월 14일에 아래와 같은 내용의 「조난」에 관한 사과문을 공개한 바 있습니다.

				\subsection{사과드립니다.}
				금치산자레시피의 김미루입니다. 며칠 전 저는 2018년 10월에 올라온 @B02Bot에 관한 트윗을 읽게 되었습니다. 트윗을 작성한 사용자는 위의 문장에 대해 실제 여성 편의점 노동자의 입장에서 무척 끔찍하다고 호소했습니다. 지금이 9월이니 거의 1년 만에 그 트윗을 보게된 셈입니다. 이 글은 그 트윗에 대한 응답입니다. 응답이 늦어진 점 사과드립니다.

『B02』는 2009년에 종이책으로 100부 이하로 소량 발행되었고, 이후 금치산자레시피를 통해 2016년에 이북으로 배포되기 시작했습니다. 현재는 운영이 중지된 @B02Bot은 종이책 『B02』가 발행되고 얼마 안있어 운영이 시작되었고, 제가 위에 언급한 트윗을 읽기 전까지 방치되어 있었습니다. 위 캡쳐된 트윗은 @B02bot이 비정기적으로 타임라인에 올리던 트윗 중 하나로 『B02』에 수록된 「조난」에 포함된 문장입니다. 

제가 「조난」의 문장들이 여성 노동자들에게 큰 상처가 될 것이라는 점을 모르고 있었던 건 아닙니다. 하지만 상처 입히기 위해 쓴 문장이 결코 아니고, 혹시라도 상처를 주게 된다고 해도 그 영향이 미미할 것이라 감히 판단했습니다. 그리고 이런 위악적 표현들이 글의 의도를 선명하게 드러낼 수 있으리라 기대했습니다. 지금은 그때의 판단과 입장에 결코 동의하지 않습니다. 의도니 뭐니 하는 것들은 죄다 헛소리입니다. 저는 이따위로 머리 굴리는 모든 일들이 비열한 개수작이라고 보고 있습니다. 저는 제가 「조난」에서 쓴 문장들이, 특히 여성 편의점 노동자들에겐 실질적인 위협이 될 수 있다는 사실을 결코 간과해선 안되었습니다.

좋은 의도는 좋은 방식으로 드러나야만 합니다. 최소한 저는 좋은 의도를 드러내기 위해 악을 자처해선 안된다고 생각합니다. 때문에 제가 작성한 「조난」의 문장들로 인해 상처 받은 분들과 위협을 느낀 분들에게, 특히 여성 편의점 노동자들에게 사과드립니다.

이 시간부터 금치산자레시피를 통해 공개되는 『B02』의 소개, 책의 「조난」이 시작되는 부분에 오늘 공개된 사과문과 「조난」의 독자들을 자극할 수 있는 요소에 대한 트리거워닝이 포함될 것입니다. 또한 앞으로 공개될 모든 저작물의 공개에 앞서 소수자에 대한 위협의 가능성을 신중히 검토하도록 하겠습니다.

저와 금치산자레시피 일동은 여성 노동자들의 삶이 항상 안전하길 바라며 이를 위한 가능한 지지와 지원을 아끼지 않겠습니다. 또한 저희 자신을 향한 감시와 의심을 멈추지 않겠습니다.

읽어주셔서 감사합니다.

\clearpage

\section{ }
이 글에는 소아 강간을 포함해 여성을 대상으로 한 성범죄 모의에 관한 언급이 포함되어 있습니다.

\section{\#} 
					%섹션 제목을 #로 설정한다. 자동화 할 수 있으나 능력이 없어 수작업 한다.
	돈을 쥔 그는 가장 먼저 편의점에 들어갔다. 담배 두 보루를 샀다. 근처 ATM에서 밀린 전기세를 냈다. 가스비를 낼 돈은 없었다. 하지만 얼마간 간신히 먹을 거리를 살 돈은 남았다. 집으로 향했다. 환한 대낮이었다. 집 안의 모든 조명을 켰다. 방 안에 형광등 울리는 소리가 가득찼다. 꺼진 핸드폰을 충전했다. 잠시 잠들었다. 꿈은 여전히 막막했다. 깜깜했다.

	만약 누군가 그들의 위치를 알고 싶어 하는 사람들이 있다면 난 기꺼이 다음 날의 절차를 따르거나 다음 날의 사건들이 일어나길 기다리라고 권할 것이다.
	
				\begin{adjustwidth}{1em}{3em}	
	\begin{enumerate}
	\item 예고되지 않은 평일 오전 일곱 시 삼십 분 서울-수도권의 모든 지하철이 일제히 폭발한다. 
	\item 아직 열기가 그치지 않은 시커먼 페허에 탄저균 분말이 비밀리에 살포된다. 
	\item 1주일 뒤 언론과 정부의 주도로 유가족 및 부상자에 대한 모금운동 및 구호활동이 시작된다. 
	\item 구호활동에 참가한 모든 사람들이 급성 탄저병으로 사망한다. 
	\item 구호활동에 참가한 모든 사람들의 친인척 및 이웃이 급성 탄저병으로 사망한다. 
	\item 그 밖의 모든 친인척 및 이웃이 급성 탄저병으로 사망한다. 
	\end{enumerate}
				\end{adjustwidth}

	6의 순서가 완료되고 한 달 뒤, 남아있는 생존자들을 수소문 하다 보면 이들을 발견할 수 있을 것이다.

	\section{\#}
	해가 뜨고, ㄴ이 끈적끈적한 얼굴로 모니터를 보고 있다. 인터넷이 끊어졌다. 예상한 일이었다. 하지만 그 때가 언제가 되더라도 그는 지금처럼 황망히 모니터를 보고 있었을 것이다. 컴퓨터가 꺼졌다. 전기가 끊어졌다. 그는 꺼진 모니터를 보다 일어났다. 화장실로 향했다. 이미 그가 모르는 사이 가스가 끊어졌다. 발가벗은 ㄴ이 차가운 샤워기 아래 서 있었다.

	그는 이번 일이 아니더라도 여러 차례 이런 일을 겪었다. 그때마다 힘없이 원망했지만 부당한 처사라는 생각은 안했다. 그들의 집행은 정당했고 그는 그 집행을 전적으로 신뢰해왔다. 물론 조금만 봐준다면 좋겠지만, 만약 그들이 그렇게 해주었다 하더라도 그가 어떻게 해볼 수 있는 일은 아무것도 없었다. 물론 조금만 봐준다면 좋겠지만, 그럼에도 그가 어떻게 해볼 수 있는 것은 아무 것도 없었다. 이때 그의 원망은 어딜 향하고 있었을까.

	내가 너무 주절주절 떠드는 것이 아닌가 겁이 난다.

	ㄴ이 매트리스에 누워 천장을 봤다. 만약 그 곳이 천장이 맞다면 말이다. ㄴ이 몸을 돌려 방문을 봤다. 만약 그 곳이 방문이 맞다면 말이다. ㄴ이 몸을 일으켜 창문을 봤다. 만약 그 곳이 창문이 맞다면 말이다. ㄴ이 팔을 뻗는다. 만약 그것이 ㄴ의 팔이 맞다면 말이다. ㄴ의 방 안이 맞다면 말이다. 

	ㄴ의 엠피쓰리 플레이어가 꺼졌다. ㄴ의 핸드폰이 꺼졌다. 이제 완전히 어둠이다. 어둠이 닿는다. ㄴ의 다리 근처에 자리 잡은 어둠은 메트리스와 맞닿은 벽을 타고 천장으로 기어올라간다. 점도 높은 액체처럼 그를 감싼다. ㄴ이 떠밀려 간다. 언젠가 아침이 오리라는 자명한 일이 아득하다. ㄴ의 좁은 방 안을 가득 채운다. 만약 그곳이 ㄴ의 좁은 방 안이 맞다면 말이다.

	\section{\#}
	하나. 검은 색이 아니었다. 둘. 짙은 파랑은 더더욱 아니었다. 그리고 거기 뭔가 있었다. 어둠은 얇은 막 같은 것이었다. 그 막이 나를 얇게 덮었다. 그 막의 바깥에는 멀쩡한 천장이 있었다거나 멀쩡한 책장이 있었다거나, 뭐 그런건 아니다. 그냥 그 얇은 막을 떨쳐낼 수 있을 만한 성질의 무엇이 아니었다. 셋. 내가 너무 주절주절 떠드는 것이 아닌가 싶어 겁이 난다. 넷. 분명한 건, 내 방 안에는 약 백아흔 뼘의 어둠이 있었다는 것이다. 

	\section{\#}
	그가 말했다. 
	
	마지막으로 그녀가 내게 전화했고, 전기가 끊어졌지. 그 년이 대뜸 내게 전화로 한다는 말이 고객님 이번에도 돈 안내면 전기가 끊어진다고, 등신 새끼야. 내가 몇 번이나 전화를 걸었니. 남들이 보면 우리 연애하는 줄 알겠다, 그치. 이러고 자빠졌네. 근데 돈이 없는걸 어쩌라고. 너도 알잖아, 씨발 돈이 없는데 내가 무얼 어쩌라고. 그래서 나는 한결같이 네 년을 사랑한다고 말하고 끊었지.

	밤이 오고, 전기가 끊어져 불도 켜지지 않는 지하 방에 누워서, 내가 보던 곳이 천장이 맞다면, 천장을 보면서 생각했어. 그 씨발년도 아마 언젠가는 전기가 끊어질 거란 말이지. 뭐 한전이 존나 관대해. 나 같으면 하루만 밀려도 바로 뒤통수를 쪼개면서 전기를 끊었을거란 말야. 근데 이건 뭐 존나 관대하잖아. 내가 세 달이나 돈을 안 냈는데 전기를 안 끊네. 이건 뭐 부처가 옆 집에서 딸딸이를 치면서 환생할 일이지. 그 씨발년이 전기가 끊어질 때 쯤 나한테 했던 것처럼 그 년도 언젠가는 다른 씨발년에게 전화를 받을거란 말이야. 고객님 이번에도 돈 안내면 전기 끊어진다고. 그리고 그 씨발년도 밤을 맞이하고 전기가 끊어져 불도 켜지지 않는 지하 방에 누워서 천장을 보면서 생각하겠지.

	그가 일어나 계단 위 미니스톱에 들어간다. 못생긴 아르바이트가 카운터에서 졸고 있다. 그는 가장 안쪽 ATM에 자리를 잡고 서 있다. 다음 날 ㄴ이 미니스톱에 들어간다. 못생긴 아르바이트가 졸린 눈으로 일어나 ㄴ을 쳐다본다. 고개를 돌려 안쪽 ATM에 그가 서 있음을 확인한다. ㄴ이 알바에게 조용히 말한다. 씨발년, 돈 내놔.

	우리 둘이 저 앞에 보이는 미니스톱에 쳐들어갔어. 불쌍한 알바 년 젖통을 꼴아보면서 말했어. 씨발년 돈 내놔. 그 년이 전기세를 꼬박꼬박 낼 정도로 대가리 좀 굴리고 현명한 년이라면 자기 지갑을 털어서 돈을 줄까? 아니면 돈통을 열어서 돈을 줄까? 

	\section{\#}
	아침이 와야 잠든다. 그때 쯤 되어야 단단한 어둠이 녹아내리기 시작한다. 밤에 잠이 안온다. 이렇게 몇 일이 지난다. 별 수 없이 집 밖으로 나간다. 처음엔 가로등 근처 벤치에 앉아 있는다. 숨도 못쉬다 물에 빠져 죽은 사체처럼 새벽의 끝자락을 따라 벤치에서 떠밀려 나온다. 어떤 사람들은 이 시점에서 아예 어둠과 하나가 된다. ㄴ으로선 그게 어떤 순간인지, 어떤 세계인지 상상하긴 힘들다. 다만 천장을 뚫어지게 응시하다 누군가의 왼쪽 한켠에 그럴 가능성이 수줍게 앉아 있을 수도 있겠다는 생각은 했다. 아직도 ㄴ은 자신이 어둠과 하나가 될 수 없었던 것이 용기가 없었기 때문이라고 생각한다. 

	다시 벤치로 돌아간다. 그 전 처럼 해가 뜰 때까지 거기 앉아 있거나 서 있는다. 가끔 운이 좋으면 근처에 떨어진 장초 몇 개를 발견한다. 게다가 더 운이 좋으면 아직 기름이 남아 있는 라이터를 구할 수도 있다. 점점 요령이 생긴다. 장초를 주워다 핀다. 그의 장초 지도가 만들어지기 시작한다. 점점 새벽에 나와 할 수 있는 일들이 많아진다.
	
	\section{\#}
	그 누구야, 김장훈이는 지가 공연해서 번 돈은 죄다 기부로 때려 박는다지. 근데 난 여태 한 번도 김장훈이 이름으로 백원 한 푼 받아본 적이 없어. 넌, 넌 받아봤어? 아니. 기부, 좆까라 그래, 한국에 그 돈 받는 사람이 한 명이라도 있을까? 아마 있지 않을까? 왜 우린 안주는데? 왜? 게으른 놈들한테는 안 주나보지. 야 말은 바로 해야지. 넌 씨발놈이다, 개 씨발놈이다. 이걸 분명히 할 필요가 있다. 우린 단 한 번도 게을러 본 적이 없다. 그냥 우리가 한 짓이 돈이 안 되었던거야. 그러니까 뭐야, 그래. 이 산업사회에서 생산성이 없었다는거지. 애초에 생산성이란 말이 왜 생겼겠냐. 증기기관 전에 생산성이란 말을 쓰기나 했겠니. 안 썼어. 그딴 말 없었어. 존나 대량생산하면서, 공정을 단순화하기 위해 계량하고 측정하면서 나온 말이란 말이야. 이런 차원에서 우리가 싸돌아다니면서 장초 줍는 건 절대로 생산성 있는 일이 아니지. 왜냐면 우린 씨발 석유를 안 쓰거든! 

	우리가 이곳 저곳의 잇플레이스들을 개척하면서 최고의 장초를 지퍼백에 담는 것은 절대로 산업사회에서 생산성 있는 일이라고 볼 수 없을거야. 그렇다고 우리가 게으르다, 그건 또 다른 문제란 말야. 우리 존나 부지런하게 장초 주우러 다니잖아. 너, 니 인생을 탈탈 털어서 이렇게 부지런해 본 적 있어? 없지? 나도 없어! 니 말대로 백 번 양보해서 만약 김장훈이가 우리와는 달리 부지런한데 돈도 없는 것들한테 기부하는거라면, 그래서 게으르면서 돈도 없는 것들한테는 기부하지 않는거라면 저 씨발놈은 기본적으로 게으른 것이 무엇인지 말할 준비가 되어 있겠지? 그리고 부지런한게 어떤 것인지 말할 준비가 되어 있겠지? 게으른 놈들이 어떤 놈들인지 말할 자신이 있겠지? 부지런한 놈이 어떤 놈들인지 말할 자신도 있겠지?

	한정품에 신상이기까지 하면 시발 가격 장난 아니다. 사람들이 왜 신상에 한정품을 사려고 하겠니? 그냥 이뻐서? 그냥 이뻐서 사는 거라면 그건 병신인거야. 그건 지 눈이 동태 눈이란거 폭로하는거야. 그럼 뭐냐? 타인에게 인정 받고 자기만족까지 획득하려는 것이잖아! 이건 뭐 보통 일이 아냐. 

	기부도 마찬가지야. 그거 기부한다고 몇 사람이나 살겠니. 오프라 윈프리 그 검둥이 씨발년이 달러를 그렇게 기부에 때려박았는데도 난 동전 하나 받아본 적이 없어. 그래, 진짜 니 말대로 백 번 양보해서 진짜 지들이 부지런한데 가난한 사람, 배고픈데 뒤져가는 사람 살리려면 기부로 될 일이 아니잖아. 이건 뭐 사회를 때려 바꿔야지. 사회에 문제가 없다면 왜 가난한 사람들이 부지런한데도 계속 가난하냐고. 

	우리랑 마찬가지잖아. 돈이 안되는 짓을 하니까 가난한거 아냐. 자, 그럼 돈이 되는 짓은 누가 결정하는데. 그리고 그게 어떤 새끼든, 돈이 안되는 짓 했다고 가난해야 하냐고. 

	근데 김장훈이가 지 입으로 저런걸 고백할까. 못할거야. 결국 그 새끼는 이 좆같은 사회는 포기할 수 없다는 것이지. 근데 고생해서 욕 쳐먹어가면서, 좆같은 소리 들으면서 사회운동 안 해도, 기부하면 말야. 타인한테 인정받고 자기만족까지 얻어낼 수 있거든. 뭐 이런 마법같은 일이. 완벽하잖아.

	뭐 김장훈이가 지 입으로 그렇게 말하진 않겠지. 왜냐면 가난뱅이들도 자존심이 있거든. 안 가난한 새끼들의 자존심과 달리 가난뱅이들의 자존심에는 값이 매겨져 있다는 것이 좀 특별한 점이지. 암튼 기부한답시고 돈 때려박는 머저리같은 새끼들은 그게 씨발 신상 사는거랑 별 다를게 없다는 것 쯤은 인정할 필요가 있어. 다시 한 번 말하지만 너랑 난 존나 열심히 살고 있어, 돈은 안되지만. 

	근데 넌 누구니? ㄴ이 말했다. 그는 ㄴ이 들고 있던 장초를 빼앗아 물었다. 나? 응. 나, 그지 새끼지. 넌? 나도. 잠깐 정적이 흐른다. 아직 여름이다. 아 씨발 배고프다. 어쩌다 이러고 있냐? 집에 불이 안들어와. 가스는? 가스도. 넌? 나도. 그들은 나란히 앉아 웃었다. 담배 있냐? ㄴ이 물었다. 그는 얼마간 세계를 지켜보다 대답했다. 아니, 없어. 

	\section{\#}
	ㄴ이 미니스톱에 들어간다. 못생긴 아르바이트가 졸린 눈으로 일어나 ㄴ을 쳐다본다. 안쪽 ATM에 그가 서 있다. 못생긴 아르바이트의 시점에서 사각에 선 그는 크게 말했다. 씨발년 돈 내놔. 

	골통을 좀 돌려보자. 우리가 저 불쌍한 년 지갑을 탈탈 털었어. 아님 돈통을 그냥 통째로 들고 날랐어. 어쨌든 운이 안 좋으면 저 병신 같은 년은 짤릴거야. 돈도 못 지켰다고. 저 년은 다시 좆같은 고시원으로 돌아가서, 그 좆같은 고시원에는 창문도 없어. 아니면 지가 기어 나가겠지. 아니면 돌아버리든가. 저 년은 다시 좆같은 고시원으로 돌아가겠지.

	해커스토익 158페이지를 넘길 때 쯤 알바생ㅁ은 잠시 카운터에서 나와 볼펜을 꺼내들었다. 스스로 바코드를 찍었다. 캐릭터 가득 그려진 레자 지갑에서 돈을 꺼내 계산했다. 거스름돈을 챙겼다. 이제 막 새벽 두 시가 넘었다. 앞으로 풀어야 할 문제는 산더미 같다. 

	뒷모습이 허름한 남자가 가게 앞에서 담배를 피고 있다. 다른 남자가 위태로운 걸음으로 가게에 들어온다. 남자는 눈 아래 잠이 가득 한 알바생ㅁ을 노려보다 그의 가슴으로 눈을 돌린다. 알바생ㅁ은 그 시선을 의식하다 곁눈질하며 가게 앞에서 담배 피는 남자의 동태를 살핀다. 아주 잠시 그 남자와 눈이 마주친 것 같다. 가게에 들어온 남자가 가게 안을 한 바퀴 돈다. 남자의 시선이 천장을 향하고 있는 것 같다. CCTV를 찾고 있는 중일 것이다. 알바생ㅁ은 카운터 아래 버튼을 향해 손을 뻗는다. 남자가 다가간다. 알바생ㅁ은 눈을 아래로 내린다. 들키지 않게 남자의 손을 감시한다. 항상 문제는 손이다. 바깥의 남자가 가게 안을 지켜본다. 두 남자가 유리판 사이로 신호를 주고 받는 것 같다. 안 쪽의 남자가 알바생ㅁ에게 말한다. 에쎄 하나 주세요. 이어진 계산. 안 쪽의 남자가 담배를 들고 나간다. 바깥의 남자는 한동안 밖에서 담배를 피우다 다른 남자가 간 반대편 길로 걸어간다. 

	분명 그들은 신호를 주고 받았을 것이다. 그들이 나눈 시선에 어떤 의미가 있었을지 복기해 본다. 또 다른 신호는 없었을지, 날 어쩌려했던 건지. 그들은 왜 담배만 사서 돌아갔는지. 내가 받은 돈에 무슨 문제가 있는 건 아닌지. 돈을 적게 받았거나 위조지폐이거나 일련번호가 이어져 있거나. 알바생ㅁ은 아까 받은 돈을 꺼내 확인해 본다. 어떤 문제도 없다. 그러면 다른 문제가 있을 것이다. 그들은 아마 건물 뒤편에서, 그 전에 약속한 곳에서 만나 다른 곳에서 강도짓을 하고 있을지도 모른다. 아니면 다시 돌아와 그들의 일을 치를 지도 모른다. 카운터 아래 작은 버튼을 누르면 경찰이 바로 올 것인지, 그 동안 그들이 날 죽이면 어떻게 해야 할지. 그냥 돈만 가져간다고 해도 보통 일이 아닐 것이다. 아니면 그들이 날 죽이기 전에 추행하려 하거나, 그냥 추행만 하고 돌아갈지도 모른다, 돈도 가져가지 않고. 난 분명 돈을 꺼냈는데.

	그가 산 담배는 일종의 신호였을지도 모른다. 그건 어떤 신호였을까. 혹시 가게를 돌며 내가 보이지 않는 곳에 내 동태를 살필 도청장치를 설치한 것은 아닌지. 그래서 그들이 다시 돌아와 그들의 일을 치를 지도 모른다. 카운터 아래 작은 버튼을 누르면 경찰이 바로 올 것인지, 그 동안 그들이 날 죽이면 어떻게 해야 할지. 그냥 돈만 가져간다고 해도 보통 일이 아닐 것이다. 아니면 그들이 날 죽이기 전에 추행하려 하거나, 그냥 추행만 하고 돌아갈지도 모른다, 돈도 가져가지 않고. 그러면 난 어떻게 해야 할 것인지. 난 분명 돈을 꺼냈는데.

	그들이 도청장치로 내 일거수 일투족을 살피며 소문을 내는 건 아닌지. 날 추행하고 그 사진을 인터넷에 올리는 것은 아닌지, 어쩌면 점장은 내가 돈도 지키지 못한 갈보년이라며 그들과 합세해 인터넷에 나쁜 소문을 낼 지도 모른다. 난 그런 여자 아닌데. 난 그런 적 없는데. 난 돈을 꺼냈는데. 그나저나 그들의 얼굴이 기억나지 않는다. 

	알바생ㅁ이 이런 생각을 하고 있는 사이 다른 남자가 가게에 들어온다. 저 남자는 아까 가게 앞에서 담배 피우던 남자가 분명하다. 

	\section{\#}
	그가 편의점에서 나와 말했다. 그의 배에서 소리가 난다. 암튼 왠만해선 도둑질은 안 하는게 좋아. 그래도 도둑질을 한다고 하더라도 반드시 지켜야 할 룰이 있지.
	
	그 바닥의? 도둑놈의?
	
	아니, 인간으로서. 
	
	무슨? 
	
	인간으로서, 최소한 이 도둑질이 인류 최후의 도둑질이어야 한다는 일말의 책임감을 가져야 한다는거지. 

	그가 걷다 장초 하나를 주워 주머니의 지퍼백에 담는다. 꽤 많은 장초가 모여있다. 

	돈을 훔치면 말야, 좆되는 새끼가 분명히 있단 말이야. 그렇지. 우리가 뭐 프린터에서 돈을 뽑아내는 것도 아니고 누군가의 돈을 낼름 가져 오는거란 말이지? 그 새끼는 그 돈으로 적금을 들든 마누라 백을 사주든 카드빚을 갚든 내연녀의 집세를 대신 내주든 애새끼 학원비를 내주든 뭐든 할거란 말야. 근데 우리가 그 돈을 훔쳐왔어! 그럼 어떻게 되겠어? 그 새끼는 당분간 그 돈으로 해야 할 짓들을 못하게 되는거란 말야! 적금은 펑크나고 마누라는 백을 못 갖고 빚은 이자를 낳고 내연녀는 날 떠난다고 하고 애새끼는 씨발 무능한 아빠 때문에 대학 못 간다고 지랄할 것이고. 그러니까. 이왕 우리가 무슨 짓을 하든 좆될 것 같으면 말이지. 조금 덜 좆되게 하는 편이 좋지 않나 하는거지. 
	
	그렇지. 
	
	인본적으로다가. 이게 힘들지. 
	
	그렇지. 힘들지. 
	
	아니, 또 인본적이고 나발이고, 인간이고 나발이고 다 집어치우더라도 그냥 깔끔하게 돈만 훔쳐 오는 게 우리한테도 더 낫단 말이지. 지속 가능한 절도를 위해선 말야 되도록 살살 하는게 좋다고. 요즘 유행하잖냐 지속가능한! 아 씨발, 이 말이 여기 오기까지 존나 오래 걸렸다. 암튼 살살해야 한다고. 그래야 오래 해쳐먹는다고. 
	
	인류 최후의 도둑질은 어떻게 하고? 

	그건 좆까라고, 만약에 우리의 일이 깔끔하지 못해서 들켰어, 그럼 어떻게 해야 돼? 죽여버려야 돼. 답이 없어. 강도 새끼들이 그냥 인정사정 없어서 사람들 죽이겠니? 아니거든! 들켜서 그런거라고. 죽여본 놈은 또 죽이게 되어 있어, 죽일 수 있다는 걸 알았거든! 성장한 것이지. 돌이킬 수 없는 성장이지.

	또 어떤 새끼들은 꼭 여자 집만 들어가, 좆질에 맛들려서 돈도 훔치고 몸도 훔치는거지. 그런 집 털어봤자 얼마 안 남는다고. 그 년들이 돈이 있으면 얼마나 있겠니. 돈 있는 년들이 그렇게 털릴 집에서 살고 있겠니? 그 새끼들은 어차피 돈 떨어지면 다시 돌아갈텐데, 이런 생각인거야. 그러니 좆질이라도 해보자는 것이지. 그럴거라면 좀 멀리 봐야 한단 말이지. 괜한 원망 사봐. 인생 지대로 꼬여가는거야. 

	그나저나 성폭행을 발명한 새끼는 존나 위대한 새끼야. 뒤돌아 나가면서 ‘죽이는 것 보단 낫잖아’ 이런 변명도 늘어놓을 수 있는 훌륭한 파생상품이라고, 존나 효과적이잖냐. 게다가 사용자 친화적이고, 환경 친화적인데다가 괜히 고달픈 인생론 논하지 않더라도 상관 없단 말이지. 왜? 살았으니까. 여기서는 그냥 살아 있으면 그만이야. 숨만 붙이고 있으면 아~ 그나마 다행이다, 고마운 강간범 님. 피해자는 어서 가해자에게 감사하다고 말씀하셔요, 살려줬으니까. 이런게 가능한 나라라니까. 완벽하다. 이런 건 홍익인간의 이념으로다가 편의점에서도 팔아야 된다니까. 

	근데 난 뭐 훔칠 생각도 없고, 누굴 강간할 생각은 더 없어. 

	\section{\#}
	만약 우리가 이 근처에 있는 귀금속 가게를 턴다고 생각해보자. 뭘 하려면 제대로 해야 한단 말이야. 뭐 그래도 마약이나 총 같은건 안 훔치는게 더 나아 그런건 좀 먹고 살만한 애들이 하는거거든? 워낙 창업비용이 많이 들어요. 예를 들어서 지금 니 손에 마약 삼백 그램이 있어. 그럼 니가 그 봉다리 들고 쫄랑쫄랑 이태원 가서 요 가이즈, 드럭~ 이러면 팔리겠냐고. 그 바닥에는 자격증 같은거 없어. 기출문제집 없이는 어디서부터 시작해야 할 지 모르는 놈은 평생가도 모른다고.

	우리가 여기 탁 모여서 보석들을 늘어놓고 나눈다고 생각해봐. 그럼 얼마나 떨어질 것 같냐? 보석 그거 훔쳐다가 막 아무데나 팔 것 같니? 아니거든, 그것만 사는 새끼들이 있어. 좆같은 새끼들이 수수료를 존나 뜯어먹는다고. 근데 경찰에선 이런 거 늘어놓고 시가 오천만 원 어치라고 지랄하잖아? 병신들아, 장물이잖아! 그거 팔아야 수수료 떼고 나면 얼마 안 남는다고요. 그게 싯가로 팔리겠냐고요. 좆도 모르는 새끼들이 절도범더러 무위도식 한다고 그러는데. 좆까라 그래. 개 등신같은 새끼들. 그러니까. 

	그러니까? 
	
	괜히 맘고생 하지 말고. 국가 재산을 훔쳐. 응? 공공재를 훔치는거지. 가드레일 같은거? 에라이 바보야, 그런거 말고. 경제적으로다가 생각해보라고, 그런 건 존나 리스크가 크다니까? 대체로 감당할 수 있다면, 리스크가 큰 것도 괜찮은데.  넌 아예 안전망이 없잖냐. 친구가 있니, 애미애비가 있니. 없잖아? 그니까, 도서관을 털어. 
	
	도서관 좋다. 
	
	지식의 전당이지. 게다가 훔친다고 피해볼 사람이 있느냐? 그렇지도 않거든. 
	
	시민들.
	
	이 세상에는 시민들이 존재해본 적이 없다. 그냥 가능성만 있는거야.

	공무원들. 

	아냐, 공무원은 사람이 아니야. 사람이라면 공무원일 수 없지. 걔네는 무시해도 괜찮아. 만약 네가 성공적으로 책을 훔쳐서, 언젠가 그 책이 없어졌다는 사실이 드러날 경우. 그때 비로소 그들은 공무원에서 사람이 될 수 있는거야. 공무원들은 실수를 하면 욕을 먹지. 그리고 시말서를 써, 아니면 경위서 뭐 비슷한 것들 말야. 이때야 말로 그들은 그들이 공무원이 되기 전에 기억하고 있던, 가슴 가장 깊은 곳에 내면화된 인간적인 미덕들을 끌어내는 성스러운 순간이야.

	\section{\#}
	책은 널 생존으로 몰아세울 수 있을거다. 그렇다고 책을 읽으면 네가 생존할 수 있느냐? 그렇진 않아. 책과 생존의 관계는 말하자면 이런거지. 네가 만약 생존을 등에 업고 내일을 맞이하고, 그 다음 날 주를 맞이하고, 그 다음 날달을 맞이하고, 그 다음 날해를 맞이하려면 어떻게 해서든 책과 만나야 할 거야. 이런거야. 단순해. 

	넌 배가 고프다. 넌 밥을 먹지 않았기 때문이지. 네가 밥을 먹지 않은 건 집에 쌀이 없기 때문이지. 이 순간 넌 밥을 먹지 않은 것이 아니라 밥을 못 먹은 것이 되지. 넌 쌀을 살 수 없었어, 돈이 없으니까. 내일도 돈이 없을거야. 

	내일 모래도. 만약 내가 내일 부터 배고프지 않을 수 있는 방법이 당장 옆집 이혼녀의 어린 딸아이를 강간하는 것 밖에 없다면 예순 번이라도 그 짓을 반복할 수 있을거야. 

	지금 넌 배가 무지 고프니까. 단순해. 하지만 넌 돈이 없지. 

	그렇지. 

	넌 돈이 필요하지? 

	그렇지. 

	그럼 돈을 훔쳐. 

	단순하네. 

	너의 생존은 오로지 너의 문제야. 우리의 문제가 아니다. 만약 지나가는 어느 누굴 붙잡고, 저기요, 저의 생존은 저의 문제이지요? 댁의 문제는 절대로 아닐테지요? 라고 물어본다면, 열에 아홉은 그냥 널 피해 지나갈 것이고, 열에 하나는 너의 질문에 친절히 ‘그렇지요, 저의 문제는 아니지요’ 라고 대답해 줄거야. 근데 넌 돈이 필요하지? 

	그렇지. 

	그래야 쌀을 살 수 있지? 

	그렇지. 

	그럼 돈을 훔쳐. 

	괜찮네. 단순하네. 

	\section{\#}
	책은 널 살릴 수 있어. 그렇다고 책이 없으면 네가 죽느냐? 그런 건 아니라고. 책 없으면 살 수 없다고 말하는 종자들은 일단 사람이 아니거나, 거짓말쟁이들이거나, 진짜로 책이 없으면 뒤지는 불쌍한 새끼들이거나, 심각한 허언증에 걸린 사람들이지. 지금 당장 너에게 필요한 건 책이 아니잖아? 한 끼 밥이잖아. 그리고 나서 넌 또다른 한 끼의 밥이 필요하겠지. 
	
	굳이 이런 문제 때문에 네가 고민할 필요가 없는 순간, 최소한 네가 다음 날 주 밥까지는 넉넉히 먹을 수 있는 순간이 오면, 그 때 넌 다른 곳에 눈 돌릴 수 있겠지. 생존을 위한 독서, 조금이나마 올바른 삶, 더 맛있는 음식, 핸드드립 커피, 훌륭한 음악, 좀 더 유려한 자살 방법, 품위있는 말투, 특별한 데이트, 훌륭한 어른이 되는 방법, 형용사 가득한 위로, 도덕의 파수꾼, 상식의 대변자 같은 것들 말야. 굳이 생존을 입에 담을 필요가 없는, 돌이켜 보아 낯 뜨겁지만 그래도 보듬어 줄 수 있고, 약간의 애정도 담아줄 만한 그런 삶 말야. 하지만 넌 당장 한 끼의 밥이 필요하지. 

	그리고 많이 털 것도 없어, 니가 돈이 많이 필요한 건 아니잖아? 

	아냐 돈 많이 필요해. 
	
	에라이, 그래봤자 이삼십이면 되는거잖아?
	
	뭐 그렇지. 그러니까 한 서너 권만 훔치면 돼.
	
	서너 권 가지고 되나? 

	응. 화집, 도록. 자고로 비싼 책은 만져보면 안다. 그리고 펼쳐보면 바로 알지. 일단 글자가 별로 없어, 그리고 존나 무거워. 도서관에도 보안시설이 되어 있긴 한데, 별거 없어. 잘해봐. 
	
	\clearpage{}
	%	검은색면이 들어갈 자리
	%		홀수 페이지에서 검은색면이 시작될 경우 검은색면2는 생략할 수 있다.
	%		이 작업은 memoir에서 자동화 할 수 있으나 능력이 안되어 수작업한다. 
		\savepagenumber
		\pagestyle{empty} 
	%	이 페이지 부터 페이지 번호를 세지 않는다. 또한 페이지 번호를 표기하지 않는다. 이 페이지는 글의 연장이 아닌 잘리지 못한 종이로서 참여한다.
	\rule[0pt]{\textwidth}{\textheight}
	
%	\clearpage{}
%	검은색면2가 들어갈 자리
%	\rule[0pt]{\textwidth}{\textheight}

	\clearpage{}
	% 검은색면이 모두 끝났다. 모든 것은 이전과 같아진다.
		\restorepagenumber
		\pagestyle{plain}

	\section{\#}
	이거 세 권 해서 십오만 원 쳐줄게. 헌책방 주인이 말했다. 그가 말했던 것처럼 청계천에는 헌책방이 많았다. ㄴ이 예상했던 것 보다 더 자세히 살펴보는 통에 바짝 긴장해 있었다. 실은 ㄴ은 자기가 들고 온 책이 원래 얼마인지도 잘 몰랐다. ㄴ은 고개를 끄덕였다. 어쨌든 상관 없었다. 저 돈이면 충분했다. 더 바라면 큰일날 것 같았다. 도서관에서 돈을 훔쳐 지하철을 타고 헌책방까지 가는 과정에서 ㄴ이 했던 일련의 수상한 행동들이 경고했다. 돈을 챙겨 나서는 ㄴ에게 책방 주인이 말한다. 워낙 도둑놈들이 많아서 말야. 

	그리고 몇 달이 지났다. 난 일을 구했다. 항상 쪼들리지만, 그래도 조금 더 나아지긴 했다. 그 동안 전기도, 가스도 끊어지지 않았다. 다행인 일이다. 그냥 인터넷 요금은 내지 않기로 했다. 핸드폰 요금도 낼 필요가 없었던 것 같다. 지금은 후회하고 있다. 내 주머니엔 차가운 핸드폰이 들어 있지만, 며칠 째 이걸 꺼낼 생각이 들지 않았다. 그러다 문득 그의 전화번호가 궁금해졌다. 그 이후로 그 곳에 가본 적은 없다. 갈 이유가 없었던 것이다. 장초가 필요할 만큼 쪼들리지 않았으니까 말이다. 다행인 일이다. 

	월급날이 되었고, 난 담배 두 보루를 샀다. 오랫만에 찾아간 그곳엔 역시 그가 쭈그려 앉아 장초를 주워피고 있었다. 그에게 담배 한 보루를 건냈다. 

	\section{\#}
	그가 말했다. 

	너와 나의 시간이 다르다는 것. 너와 나의 시간은 달라. 너와 나의 시간은 항상 달랐고, 앞으로도 그럴거야. 우린 알잖아. 이게 단순히 속도가 다르다는 말은 아니라는 것. 너와 나는 다른 시간을 갖고 있다는거야. 너와 나는 다른 시간 위에 서 있다는 것이고, 너와 나에겐 다른 시간이 몰아치고 있는거야. 보다 정확히 말하면 너와 나의 시간은 질이 달라. 그가 ㄴ이 피던 장초를 들어 물었다. 조금이나마 더 분명해졌으면 좋겠다. 

	응, 나 요즘 일해. ㄴ이 말했다. 그가 잠시 바닥을 살펴본다. 멋진데? 그가 말했다. 

	ㄴ의 집이 가로로 열 다섯개는 놓일 법한 고가도로가 머리 위에 떠 있다. 그 아래 ㄴ이 걸어왔을 다리가 흔들리며 덤프트럭을 뱉어낸다. ㄴ은 어쩌면 다리 건너의 시간이 이곳과 질적으로 다를지도 모른다는 상상을 한다. 

	너나 나나. 우린. 그가 말했다. 응. 그래, 우린, 우린 알잖아. 응. 뭐 사실은. 그가 한동안 텅 빈 도로를 지켜본다. ㄴ은 그의 옆으로 가까이 다가가 말했다. 나 요즘 일해. 

	그래, 멋있다 씨발 근데 진짜 좆같은게 존나 구차하잖아, 항상 이, 이 아가리로 생존이니 뭐니 궁시렁 거리는게 얼마나 쪽팔리고, 구차하고, 씨발 존나 짜증나지 않냐? 응. 그가 ㄴ을 보며 말한다. 난 잘 모르겠다, 우린 내일도, 어쩌면 내년에도 이런 구차하고 좆같은 소리나 하고 있는게 아닐까, 우린 평생 이러는 건 아닐까. 뭐 그냥 씨발 좆같잖아, 지금도. 

	응. 나 요즘 일해.

	\section{\#}
	흔들리던 가로등이 쓰러지고, 죽은 바이크를 토해낸다. 죽은 바이크의 사체 위로 물기 어린 나트륨등이 깨어진다. ㄴ이 기억하는 그 가로등의 역사상 가장 밝은 빛이었다. ㄴ과 그는 아주 잠시동안 망막 한켠엔 죽은 바이크와 가로등의 마지막 빛의 자리를 마련하고 고개를 돌린다. ㄴ이 담배를 끄고 텅 빈 8차선 도로 중앙선을 향해 걸어간다. 우아한 걸음걸이였다. ㄴ의 집이 가로로 열 다섯개는 놓일 법한 고가 도로가 머리 위에 떠 있다. 그 아래 ㄴ이 걸어왔을 다리가 흔들리며 또 한번 덤프트럭을 뱉어낸다. 덤프트럭 한 쪽 네 바퀴가 ㄴ을 밟고 지나간다. 

	ㄴ이 그의 자리를 지켜보는 동안 서너 대의 덤프트럭이 더 지나갔다. ㄴ은 한쪽 주머니에 쑤셔놓은 지퍼백을 꺼낸다. 반투명한 지퍼백에는 ㄴ의 입술을 거친 수 백명의 타액이 새까맣게 뭉게져 있다. 아직 밟히지 않은 그의 창자 한 쪽을 뚫고 회충이 기어나온다. ㄴ은 그의 구겨진 창자와 망가진 턱과 늘어진 혀와 아직 남아있는 안구 한쪽과 깨진 가슴뼈와 아직 소화되지 않은 은박지와 찢어진 가죽을 지퍼백에 담았다. ㄴ은 아직 길게 늘어져 있는 그의 잔해를 내버려두고 이 곳으로 왔던 다리를 따라 집으로 돌아가기로 했다. 다리 아래 낡은 개천을 따라 그가 담긴 지퍼백이 흘러간다.

	\newpage{}
	\section*{}
	
\backmatter %책의 뒷부분을 시작한다. 차례와 같단한 코멘트, 그리고 권리에 대해 명시한다. 그리고 뒤표지.

	%차례를 시작한다.
	\newpage{}
	%호출할 차례가 페이지에 어떻게 자리잡아야 할 것인지 설정한다.
		\pagestyle{empty}	
		\maxtocdepth{chapter} 
		%chapter까지만 차례에 포함한다. 이보다 낮은 계층의 section은 차례에 포함하지 않는다.
	\tableofcontents*{}
	%차례를 호출한다.
	
	\chapter*{끝}
	%코멘트
	\begin{itemize}
	\item{석고보드} | 2009년 여름에 완성했다.
	 
	 \item{유예된 추락} | 2008년 겨울에 완성했다. 이 겨울이 끝나고 2008년 봄이 왔다.
	
	\item{장마} | 2007년 여름에 완성했다. 한페이지단편소설\footnote{http://www.1pagestory.com/}에서 출간한 한단설 베스트 2007\footnote{이 책은 한페이지단편소설 웹사이트에서 살 수 있다.}에 수록되어 있다. B02에 실린 장마는 2009년에 새로 써 다시 완성한 글이다.

	\item{조난} | 2010년 여름에 완성했다.
	
	\item{발인} | 2010년 봄에 완성했다.
	\end{itemize}

	\newpage{}
	%권리와 책 정보
	\calccentering{\unitlength}
	\begin{adjustwidth*}{7em}{1em}
	\begin{description}
	\item{\textbf{발행일}}
		2010년 12월
	\item{\textbf{지은이}} 
		미루 mirupsec@gmail.com
	\item{\textbf{웹사이트}}
		http://joeaney.tumblr.com
	\item{\textbf{책임과 저작권}}
		이 책에 포함된 모든 저작물은 \emph{크리에이티브 커먼즈 저작자표시-동일조건변경허락 2.0 대한민국 라이선스}에 따라 이용할 수 있다. 
	\end{description}
	\end{adjustwidth*}
	
	\newpage{}
	%뒤표지. 권리와 책 제목을 표기한다.
	2010년
	
	비공이
	
	미루
	
	\vspace{\stretch{1}}
	\begin{center}
	\scriptsize{크리에이티브 커먼즈 저작자표시-동일조건변경허락 2.0 대한민국 라이선스}\pagebreak
	\end{center}

\end{document}
%책이 모두 끝났음을 선언했다.